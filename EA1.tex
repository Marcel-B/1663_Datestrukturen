\documentclass[12pt, twoside]{article}
\usepackage[utf8]{inputenc}
\usepackage[T1]{fontenc}
\usepackage{lmodern}
\usepackage{ngerman}
\usepackage{amsmath}
\usepackage{amssymb}
\usepackage{array}
\setlength{\parindent}{0em} 
\newcommand{\N}{\mathbb{N}}
\newcommand{\M}{$\times$}
\newtheorem{Sa}{Satz}[subsection]
\newtheorem{Kor}{Korollar}[subsection]
\newtheorem{Prop}{Proposition}[subsection]
\newtheorem{Def}{Definition}[subsection]
\begin{document}


\section{Aufgabe 2}
\begin{tabbing}
xxxxxxxx \= xxxxxxxxxxxx \= xxxxxxxxxxxxxxxx \=   xxxxxxxxxxxxxxxxxxxxx  \kill
Teil (a) \\
algebra \> rational          \\
sorts       \> rational, int, bool                                                  \\
ops         \> add:                 \> rational$\times$rational   \> $\rightarrow$ rational     \\
            \> multi:               \> rational$\times$rational   \> $\rightarrow$ rational     \\
            \> rproc:               \> rational                   \> $\rightarrow$ rational     \\
            \> isequal:             \> rational$\times$rational   \> $\rightarrow$ bool         \\
            \> frac:                \> int$\times$int             \> $\rightarrow$ rational     \\
            \> nom:                 \> rational                   \> $\rightarrow$ int          \\
            \> denom:               \> rational                   \> $\rightarrow$ int          \\
            \> num2frac:            \> int                        \> $\rightarrow$ rational     \\
            \> frac2num:            \> rational                   \> $\rightarrow$ int          \\
            \> isnum:               \> rational                   \> $\rightarrow$ bool         \\
            \\
Teil (b) \\
sets        \> bool = \{true, false\} \\
            \> int = $\mathbb{Z}$ \\
            \> rational = $\{(a,b)|a \in \text{int}, b \in \text{int} \backslash \{0\}\}$       \\
     
functions   \> $multi((x_1, y_1), (x_2, y_2)) = (x_1 \cdot x_2, y_1 \cdot y_2)$                 \\
            \> $add((x_1, y_1), (x_2, y_2)) = (x_1 \cdot y_2 + x_2 \cdot y_1, y_1 \cdot y_2)$   \\
            \> $rproc((x_1, y_1))=\begin{cases}
            (x_1, y_1) & \text{falls } x_1 \not = 0 \\
            \text{undefiniert} & \text{sonst}            
            \end{cases}$ \\
            \> $isequal((x_1, y_1), (x_2, y_2)) = 
            \begin{cases}
            true & x_1 \cdot y_2 = x_2 \cdot y_1 \\
            false & \text{sonst}
            \end{cases}$ \\
            \> $frac(x_1, x_2) = \begin{cases}
            (x_1, x_2) & \text{falls } x_2 \not = 0 \\
            \text{undefiniert} & \text{sonst}
            \end{cases}$ \\
            \> $denom((x_1, y_1)) = y_1$ \\
            \> $nom((x_1, y_1)) = x_1$ \\
            \> $num2frac(x_1) = (x_1, 1)$ \\
            \> $frac2num((x_1, y_1)) = \begin{cases}
            \frac{x_1}{y_1} & \text{falls }  x_1  \text{ mod }  y_1 = 0 \\
            \text{undefiniert}     & \text{sonst}
            \end{cases}$ \\
            \> $isnum((x_1, y_1)) = \begin{cases}
            true & \text{falls } x_1 \text{ mod } y_1 = 0 \\
            false & \text{sonst}
            \end{cases}
            $ 
 \end{tabbing}

 Teil (c)
\begin{enumerate}
\item $add(num2frac(x_1), num2frac(x_2)) = (x_1+x_2, 1)$ 
\item $nom(num2frac(x_1)) = x_1$ 
\item $denom(num2frac(x_1)) = 1$
\item $frac2num(num2frac(x_1)) = x_1$
\item $isnum(num2frac(x_1)) = true$
\item $multi(x, y) = multi(y, x)$
\end{enumerate}

\newpage

\section{Aufgabe 4}
$T_1(n) = 1+n \cdot 1 = 1 + n = O(n)$ \\
$T_2(n) = 1 + n \cdot (1+1+n) ) n^2+2n+n = O(n^2)$ \\
$T_3(n) = 1 + 2n + \sum_{i=1}^{n}i = O(n^2)$ 

\section{Aufgabe 3}
\begin{tabbing}
xxxxxxxx \= xxxxxxxxxxxxxxx \= xxxxxxxxxxxxxxxxxxxxxxx       \=   xxxxxxxxxxxxxxxxxxxxx  \kill
algebra  \> markt        \\
sorts    \> markt, stand, position, angebot, einheit, bool, int, real, string                             \\
ops      \> neuerMarktplatz:\> string\M int\M int\M position \> $\rightarrow$ markt        \\
         \> neuerStand:     \> rational$\times$rational   \> $\rightarrow$ stand        \\
         \> fügeStandHinzu: \> rational                   \> $\rightarrow$ markt        \\
         \> entferneStand:  \> rational$\times$rational   \> $\rightarrow$ markt        \\
         \> änderePreis:    \> int$\times$int             \> $\rightarrow$ stand        \\
         \> verkaufe:       \> rational                   \> $\rightarrow$ int          \\
         \> ergänzeAngebot: \> rational                   \> $\rightarrow$ int          \\
         \> erhöheVorrat:   \> int                        \> $\rightarrow$ stand        \\
         \> gesamterUmsatz: \> rational                   \> $\rightarrow$ int          \\
         \> Abstand:        \> rational                   \> $\rightarrow$ real         \\
         \> erreichbar:     \> \>$\rightarrow$ bool \\
sets     \> $markt = \{(name, l, b, eingang, \{st_1, st_2,\dots,st_n\}) | name \in \text{string}, l, b \in \text{int},$ \\
\> \> $st_1, st_2, \dots,st_n \in stand, eingang \in \text{position}\}$ \\
            \> $stand = \{(nr, kat, p, \{a_1, a_2,\dots, a_n\}, umsatz) | nr, umsatz \in \text{int}, kat\in \text{kategorie},$ \\
            \>  \>$ p\in \text{position}, a_1, a_2,\dots, a_n\in \text{angebot}\}$\\
            \> $position = \{(x, y) | x, y \in \text{int}\}$ \\
            \> $angebot = \{(menge, preis, ware, e) | menge, preis \in \text{int}, ware \in \text{string}, e \in \text{einheit} \}$ \\
            \> $einheit = \{obst/gemüse, fleisch, fisch, eier, käse, brot\}$ \\
 
     
functions   \> $neuerMarktplatz(name, l, b, position) = (n, l, b, (x, y))$                 \\

\end{tabbing}

\end{document}
