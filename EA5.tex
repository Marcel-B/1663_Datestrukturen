\documentclass[12pt]{article}
\usepackage[utf8]{inputenc}
\usepackage[T1]{fontenc}
\usepackage{lmodern}
\usepackage{ngerman}
\usepackage{amsmath}
\usepackage{amssymb}
\usepackage{array}
\usepackage{german}
\usepackage{tikz}
\usepackage{subfigure} 
\usepackage{fancyhdr}
\usepackage{eurosym}
\usepackage[left=2cm, right=2cm, top=3cm, bottom=2cm]{geometry}
\usepackage{listings}
\usepackage{hyperref}
\newcommand{\N}{\mathbb{N}}
\newcommand{\R}{\mathbb{R}}
\newcommand{\Z}{\mathbb{Z}}
\newcommand{\M}{$\times$}
\newcommand{\rA}{$\rightarrow $ }
\lstset{language=Pascal}
\lstset{backgroundcolor = \color{lightgray}, xleftmargin = 0.5cm, framexleftmargin = 1em}
\pagestyle{fancy}
\setlength{\parindent}{0em}
\newtheorem{Sa}{Satz}[subsection]
\newtheorem{Kor}{Korollar}[subsection]
\newtheorem{Prop}{Proposition}[subsection]
\newtheorem{Def}{Definition}[subsection]
\lhead{Marcel Benders}
\rhead{Matrikelnummer: 5431760}
\cfoot{Datenstrukturen - KE 5}
\begin{document}

\section*{EA5 - Aufgabe 1}

\subsection*{a}
\begin{figure}[h]
	\centering
	\scalebox{.5}{\input{ea5_2_abb1}}
	\caption{Ergebnisbaum}
	\label{img:abb2}
\end{figure}

\subsection*{b}
Diese Laufzeit ist möglich, da es die Schranke $k << n$ gibt. Die untere Schranke $\Omega (n  \cdot \text{log}(n))$ gilt für nicht eingeschränkte Sortierverfahren.

\section*{EA5 - Aufgabe 2}

\subsection*{a}

\subsection*{b}
Diese Laufzeit ist möglich, da es die Schranke $k << n$ gibt. Die untere Schranke $\Omega (n  \cdot \text{log}(n))$ gilt für nicht eingeschränkte Sortierverfahren.



\end{document}