\documentclass[12pt]{article}
\usepackage[utf8]{inputenc}
\usepackage[T1]{fontenc}
\usepackage{lmodern}
\usepackage{ngerman}
\usepackage{amsmath}
\usepackage{amssymb}
\usepackage{array}
\usepackage{german}
\usepackage{tikz}
\usepackage{fancyhdr}
\usepackage{eurosym}
\usepackage[left=2cm, right=2cm, top=3cm, bottom=2cm]{geometry}
\usepackage{listings}
\newcommand{\N}{\mathbb{N}}
\newcommand{\R}{\mathbb{R}}
\newcommand{\Z}{\mathbb{Z}}
\newcommand{\M}{$\times$}
\newcommand{\rA}{$\rightarrow $ }
\lstset{language=C++}
\lstset{backgroundcolor = \color{lightgray}, xleftmargin = 0.5cm, framexleftmargin = 1em}
\pagestyle{fancy}
\setlength{\parindent}{0em}
\newtheorem{Sa}{Satz}[subsection]
\newtheorem{Kor}{Korollar}[subsection]
\newtheorem{Prop}{Proposition}[subsection]
\newtheorem{Def}{Definition}[subsection]
\lhead{Marcel Benders}
\rhead{Matrikelnummer: 5431760}
\cfoot{Datenstrukturen - KE 2}
\begin{document}

\section*{EA2 - Aufgabe 1}
\section*{EA2 - Aufgabe 2}
\section*{EA2 - Aufgabe 3}
\section*{EA2 - Aufgabe 4 - Priority Queue} 
\begin{tabbing}
xxxxxxxxx \= xxxxxxxxxxxx \= xxxxxxxxxxxxxxxx \= xxxxxxxx \= \kill
\textbf{algebra} \> pqueue \\
\textbf{sorts}  \> pqueue pqelem elem priority bool \\
\textbf{ops}    \> empty        \> :                        \> \rA  pqueue \\
                \> front        \> : pqueue                 \> \rA elem \\
                \> enqueue      \> : pqueue\M pqelem    \> \rA pqueue \\ 
                \> dequeue      \> : pqueue                 \> \rA pqueue \\
                \> isempty      \> : pqueue                 \> \rA bool \\
                \> createpqelem \> : priority\M elem        \> \rA pqelem \\
\textbf{sets}   \> pqueue \> $= \{<a_1, \dots , a_n>|n \ge 0, a_i = (p_i, e_i) \in pqelem, p_i + 1 \le p_i, i \in \{1, \dots, n\} \}$ \\
                \> pqelem \> $= \{(p_i, q_i)|i \ge 0, i < n\}   = priority \times elem $ \\
                \> priority \> = $p$ \\
\textbf{functions} \> empty \>                      \> = <> \\
                \> front($<a_1, \dots , a_n>$) \>   \> = $\begin{cases} e_1 & 3 \\ e_2 & 4 \end{cases}$ \\
\end{tabbing}

\section*{EA2 - Aufgabe 1}

%\begin{figure}[h]
%	\centering
%	\scalebox{.5}{\input{EA2_Abb1}}
%	\caption{links: $P_3$, rechts $P_2$}
%	\label{img:grafik-du}
%\end{figure}

%\begin{tabbing}
%xxx \= xxx \= xxx \= xxx \= xxx \= xxx \= xxx \= xxx \= xxxxxxxxxxxxx\kill
%6   \> 6   \> 4   \> 3   \> 3   \> 2   \> 1   \> 1   \\
%    \> -1  \> -1  \> -1  \> -1  \> -1  \> -1  \> 0  \> sub \\
%    \> 5   \> 3   \> 2   \> 2   \> 1   \> 0   \> 1  \> sort \\
%    \> 5   \> 3   \> 2   \> 2   \> 1   \> 1   \> 0  \>  \\
%    \>     \> -1  \> -1  \> -1  \> -1  \> -1  \> 0  \>  sub\\
%    \>     \> 2   \> 1   \> 1   \> 0   \> 0   \> 0  \> \\
%    \>     \>     \> -1  \> -1  \> 0   \> 0   \> 0  \> sub\\
%    \>     \>     \> 0   \> 0   \> 0   \> 0   \> 0  \>  \\
% \end{tabbing}

\end{document}
