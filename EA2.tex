\documentclass[12pt]{article}
\usepackage[utf8]{inputenc}
\usepackage[T1]{fontenc}
\usepackage{lmodern}
\usepackage{ngerman}
\usepackage{amsmath}
\usepackage{amssymb}
\usepackage{array}
\usepackage{german}
\usepackage{tikz}
\usepackage{fancyhdr}
\usepackage{eurosym}
\usepackage[left=2cm, right=2cm, top=3cm, bottom=2cm]{geometry}
\usepackage{listings}
\newcommand{\N}{\mathbb{N}}
\newcommand{\R}{\mathbb{R}}
\newcommand{\Z}{\mathbb{Z}}
\newcommand{\M}{$\times$}
\newcommand{\rA}{$\rightarrow $ }
\lstset{language=Pascal}
\lstset{backgroundcolor = \color{lightgray}, xleftmargin = 0.5cm, framexleftmargin = 1em}
\pagestyle{fancy}
\setlength{\parindent}{0em}
\newtheorem{Sa}{Satz}[subsection]
\newtheorem{Kor}{Korollar}[subsection]
\newtheorem{Prop}{Proposition}[subsection]
\newtheorem{Def}{Definition}[subsection]
\lhead{Marcel Benders}
\rhead{Matrikelnummer: 5431760}
\cfoot{Datenstrukturen - KE 2}
\begin{document}

\section*{EA2 - Aufgabe 1}
\subsection*{a - Auswahl der Datenstrukturen}
\begin{itemize}
\item \textbf{Eingabefolge} - Hier kann eine \textit{Queue} verwendet werden. Die Reihenfolge bleibt erhalten, man kann nur von vorne zugreifen. 
Die kann auch u.U. Fehler vermeiden. Die Laufzeit ist mit $O(1)$ zu verzeichnen.
\item \textbf{Übersicht} - Hier bietet sich ein zweidimensionales \textit{Array} an. Dieses kann wie eine Tabelle angeordnet werden, 
Dimension 1 für die Zeilen (terminal), Dimension 2 für die Spalten (nicht terminal). In den Zellen sind dann Zeiger auf Regeln.  Die Laufzeit ist mit $O(1)$ zu verzeichnen.
\item \textbf{Regeln} - Dafür bietet sich eine \textit{doppelt verkettete Liste} an. Dies hat den Vorteil, das die Regeln von vorne und von hinten abgearbeitet werden können.
\item \textbf{Zustand Regelabarbeitung} - Die wird mit einem \textit{Stack} realisiert. Hier kann der letzte Zustand immer festgehalten werden und bei Bedarf sofort
 zugegriffen werden. Ist der Stack leer, sind die Regeln abgearbeitet worden. Auch hier gitl $O(1)$.
\end{itemize}

\subsection*{b - Verwendete Operationen}
\begin{itemize}
\item Mit der Operation \textit{empty} lassen sich Queue, doppelt verkettete Liste und Stack erzeugen. Das zweidimensionale Array muss \textit{initialisiert} werden. 
\item Mit \textit{enqueue} wird die Zeichenfolge in die Queue gegeben, mit \textit{front} kann auf das erste Element zugegriffen werden, mit \textit{dequeue} wird es dann entfernt.
\item Das Array wird über die \textit{Indizes} kontrolliert. Gibt es für eine Kombination von terminal und nicht termial keine Regel, gibt es eine Fehlermeldung.
\item Bei der verkettenten Liste werden die Regeln mit \textit{insert} hinzugefügt. Mit \textit{next} und \textit{previous} wird die Liste durchlaufen. Die Operation \textit{eol} wird benötigt
um festzustellen, ob man am Ende der Liste angekommen ist, \textit{bol} prüft ob man sich am Anfang der Liste befindet. 
\item Vom Stack wird die Operation \textit{push} benötigt, wenn ein Zustand festgehalten werden muss.  Dieser wird dann auf den Stack abgelegt. Mit \textit{top} bekommt man die Referenz auf den letzten
Zustand, mit \textit{pop} entfernt man ihn, wenn der Zustand abgearbeitet worden ist.  
\end{itemize}


\newpage
\section*{EA2 - Aufgabe 2 - 3d Algebra}
\begin{tabbing}
xxxxxxxxx \= xxxxxxxxxxx \= xxxxxxxxxxxxxx \= xxxxxxxx \= \kill
\textbf{algebra} \> 3d \\
\end{tabbing}

\subsection*{a}
\begin{tabbing}
xxxxxxxxx \= xxxxxxxxxxx \= xxxxxxxxxxxxxx \= xxxxxxxx \= \kill
\textbf{sorts}  \> point, triangle, triangles, vector, bool, surface, volumne, int \\
\textbf{sets}   \> point    \> $= \{(x,y,z)|x,y,z \in \R \}$ \\
                \> vector   \> = point \\
                \> triangle \> $= \{(a, b, c)|a,b,c \in \text{point }, a \not= b \not= c \not= a, \neg \text{collin}(a,b,c))   \}$\\
                \> triangles \> $= \{ T\subset \text{triangle }||T| < \infty \}$ \\
                \> surface  \> $= \{S \in \text{triangles }| \dots  \}$ \\
                \> volumne  \> $= \{V \in \text{surface }| \dots \} $
\end{tabbing}

\subsection*{b}

body $\in \{\text{volumne, surface}\}$
\begin{tabbing}
xxxxxxxxx   \= xxxxxxxxxxxx     \= xxxxxxxxxxxxxxxxxxxxx    \= xxxxxxxx \= \kill
\textbf{ops} \> collin          \> : point\M point\M point  \> \rA bool \\
            \> createPoint      \> : real\M real\M real     \> \rA point \\
            \> createTriangle   \> : point\M point\M point  \> \rA triangle $\cup \{\perp\} $   \\
            \> adjacent         \> : triangle\M triangle   \> \rA bool \\
            \> touch            \> : triangle\M triangle    \> \rA bool \\
            \> createSurface    \> : triangles              \> \rA surface $\cup \{\perp\}$     \\
            \> createVolumne    \> : triangles              \> \rA volumne $\cup \{\perp\}$          \\ 
            \> size             \> : body                   \> \rA int \\
            \> bbox             \> : body                   \> \rA point\M point \\
            \> translate        \> : body\M vector          \> \rA body \\
            \> createCube       \> : point\M point          \> \rA volumne $\cup \{\perp\}$ \\
\end{tabbing}


\subsection*{c}

\begin{tabbing}
xxxxxxxxx \= xxxxxxxxxxxxxxxxxxxxxxxxxxxxxx\= xxxxxxxx \= \kill
functions \> 
             collin($a,b,c$)                    \> \\
          \> createPoint($x, y, z$)             \> $= (x, y, z)$ \\           
          \> createTriangle($a, b, c$)          \> $= \begin{cases}\{(a, b, c) & a \not= b \not= c \not= a \wedge \neg \text{collin}(a, b, c) \\ \perp &  \text{sonst} \end{cases}$ \\        
          \> adjacent($(a_1, a_2, a_r), (b_1, b_2, b_3)$) \> $= \exists i, j, k, l \in \{1, 2, 3\}, i \not= j, k \not= l :$ \\ 
          \>                                                \>   $(a_i = b_k \wedge a_j = b_l) \vee (a_i = b_i \wedge a_j = b_k)$ \\     
          \> touch($(a_1, a_2, a_3), (b_1, b_2, b_3)$) \> $=\exists ! (i, j) \in \{1, 2, 3\}^2 : a_i = b_i$ \\
          \> size($\{t_1, \dots , t_n\}$)       \> $=n$ \\
\end{tabbing}

\newpage
\section*{EA2 - Aufgabe 3 - Datentyp für Mengen}
Zu definieren ist eine passende Implementierung für einen Datentyp \textit{Menge}, welcher eine Laufzeit von $O(1)$ hat.
\subsection*{a}
Um eine Laufzeit von $O(1)$ zu bekommen, bietet sich ein Array an. Es hat eine feste Größe von $n \in \N$, man kann jedes Element mit einem Index direkt ansprechen.
Fügt man ein Element ein, kann direkt geprüft werden ob der Platz frei ist, das gleiche gilt für die Prüfung, ob ein Element vorhanden ist. Das einzige Problem bereitet, laut 
Aufgabenstllung, das zufällige Entfernen. Hieraus kann sich, im worst case (leeres Array), eine Laufzeit von $O(n)$ ergeben.

Zu lösen ist dieses Problem mit einem \textit{Stack}. Wird ein Element in das Array gegeben, wird der Index auf dem Stack notiert. Soll ein Element aus dem Array entfernt werden,
nimmt man den Index vom Stack und enternt das dazugehörige Element aus dem Array. So ergibt sich bei remove ebenfalls eine Laufzeit von $O(1)$.

\subsection*{b}
Die passenden Algorithmen:

\begin{lstlisting}
algorithm contains(elem)
 if array[elem] = elem then
    return true
 else
    return false

algorithm insert(elem)  
  if contains(elem) = false then
      array[elem] := elem
      stack.push(elem)
      

algorithm remove()
   if stack.isempty() = false then
    elem := stack.top()
    stack.pop()
    array[elem] := null
    return elem;
\end{lstlisting}

\newpage
\section*{EA2 - Aufgabe 4 - Priority Queue} 
\begin{tabbing}
xxxxxxxxx \= xxxxxxxxxxx \= xxxxxxxxxxxxxx \= xxxxxxxx \= \kill
\textbf{algebra} \> pqueue \\
\textbf{sorts}  \> pqueue pqelem elem priority bool \\
\textbf{ops}    \> empty        \> :                        \> \rA  pqueue \\
                \> front        \> : pqueue                 \> \rA elem \\
                \> enqueue      \> : pqueue\M pqelem    \> \rA pqueue \\ 
                \> dequeue      \> : pqueue                 \> \rA pqueue \\
                \> isempty      \> : pqueue                 \> \rA bool \\
                \> createpqelem \> : priority\M elem        \> \rA pqelem \\
\textbf{sets}   \> pqueue \> $= \{<a_1, \dots , a_n>|n \ge 0, a_i = (p_i, e_i) \in pqelem, p_i + 1 \le p_i, \forall i \in \{1, \dots, n\} \}$ \\
                \> pqelem \> $= \{(p_i, q_i)|i \ge 0, i < n\}   = priority \times elem $ \\
                \> priority \> = $p$ \\
\textbf{functions} \> empty \>                      \> = <> \\
                \> front($<a_1, \dots , a_n>$) \>   \> = $\begin{cases} e_1 & a_1 = (p_1, e_n) \wedge n > 0 \\
                                                                       \text{undefiniert} &  \text{sonst} \end{cases}$ \\
                \> enqueue($<a_1, \dots , a_n>, a_x$) \>        \> = $\begin{cases} <a_x> & n = 0 \\ <a_1, \dots , a_i, a_x, a_{i+x}, \dots,  a_n>  & a_i=(p_i, e_i) \\
                                                                                & \wedge p_i \ge p_x > p_i+1 \end{cases}$ \\
                \> dequeue($<a_1, \dots , a_n>$) \> \> = $\begin{cases}<a_2, \dots , a_n> & n > 0 \\ 
                                                                    \text{undefiniert} & \text{sonst} \end{cases} $\\
                \>isempty($<a_1, \dots , a_n>$) \>  \> = $\begin{cases} true & n = 0 \\ false & \text{sonst}\end{cases}$ \\
                \>createpqelem($p, e$)\>            \> = $(p, e)$ \\
\end{tabbing}


%\begin{figure}[h]
%	\centering
%	\scalebox{.5}{\input{EA2_Abb1}}
%	\caption{links: $P_3$, rechts $P_2$}
%	\label{img:grafik-du}
%\end{figure}

%\begin{tabbing}
%xxx \= xxx \= xxx \= xxx \= xxx \= xxx \= xxx \= xxx \= xxxxxxxxxxxxx\kill
%6   \> 6   \> 4   \> 3   \> 3   \> 2   \> 1   \> 1   \\
%    \> -1  \> -1  \> -1  \> -1  \> -1  \> -1  \> 0  \> sub \\
%    \> 5   \> 3   \> 2   \> 2   \> 1   \> 0   \> 1  \> sort \\
%    \> 5   \> 3   \> 2   \> 2   \> 1   \> 1   \> 0  \>  \\
%    \>     \> -1  \> -1  \> -1  \> -1  \> -1  \> 0  \>  sub\\
%    \>     \> 2   \> 1   \> 1   \> 0   \> 0   \> 0  \> \\
%    \>     \>     \> -1  \> -1  \> 0   \> 0   \> 0  \> sub\\
%    \>     \>     \> 0   \> 0   \> 0   \> 0   \> 0  \>  \\
% \end{tabbing}

\end{document}
