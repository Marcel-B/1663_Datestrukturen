\documentclass[12pt]{article}
\usepackage[utf8]{inputenc}
\usepackage[T1]{fontenc}
\usepackage{lmodern}
\usepackage{ngerman}
\usepackage{amsmath}
\usepackage{amssymb}
\usepackage{array}
\usepackage{german}
\usepackage{tikz}
\usepackage{fancyhdr}
\usepackage{eurosym}
\usepackage[left=2cm, right=2cm, top=3cm, bottom=2cm]{geometry}
\usepackage{listings}
\newcommand{\N}{\mathbb{N}}
\newcommand{\R}{\mathbb{R}}
\newcommand{\Z}{\mathbb{Z}}
\newcommand{\M}{$\times$}
\newcommand{\rA}{$\rightarrow $ }
\lstset{language=Pascal}
\lstset{backgroundcolor = \color{lightgray}, xleftmargin = 0.5cm, framexleftmargin = 1em}
\pagestyle{fancy}
\setlength{\parindent}{0em}
\newtheorem{Sa}{Satz}[subsection]
\newtheorem{Kor}{Korollar}[subsection]
\newtheorem{Prop}{Proposition}[subsection]
\newtheorem{Def}{Definition}[subsection]
\lhead{Marcel Benders}
\rhead{Matrikelnummer: 5431760}
\cfoot{Datenstrukturen - KE 3}
\begin{document}

\section*{EA3 - Aufgabe 1}

\section*{EA3 - Aufgabe 3}
Als erstes nummerieren wir zur Hilfe die Buchstaben durch.

(A, 1)(B, 2)(C, 3)(D, 4)(E, 5)(F, 6)(G, 7)(H, 8)(I, 9)(J, 10)(K, 11)(L, 12)(M, 13)(N, 14)(O, 15)(P, 16)(Q, 17)(R, 18)(S, 19)(T, 20)(U, 21)(V, 22)(W, 23)(X, 24)(Y, 25)(Z, 26)
\\

Heidrun soll eingepflegt werden. Dazu werden die Werte der ersten drei Buchstaben summiert: Es gilt H=8, E=5, I=9 $\Rightarrow 8 + 5 + 9 = 22$. Daraus wird das Quadrat gebildet:
$22^2=484$. Wir nehmen den mittleren Wert, also 8. An dieser Stelle wird der Name eingepflegt. Analog werden die Plätze der anderen Namen ermittelt. Gibt es eine Kollision,
wird $i^2, \text{ mit } i = \text{ Anzahl der Kollisionen}$ zum mittleren Wert addiert. Hat der Name keinen mittleren Buchstaben, wird vom mittleren Paar der rechte
Buchstabe genommen.

Die einzelnen Schritte sind der untenstehenden Tabelle zu entnehmen:
\\
\begin{tabular}{|l|l|l|r|}
\hline
Name        & k         & $k^2$                         & $h(k)$                        \\
\hline
\hline
Heidrun     & 22        & 4\underline{8}4               & 8                             \\
Anne        & 29        & 8\underline{4}1               & 4                             \\
Hartmut     & 27        & 7\underline{2}9               & 2                             \\
Christian   & 29        & 8\underline{4}1               & $4 \Rightarrow h(x)+1^2 = 4 + 1 = 5\text{ mod }10=5$  \\
Simone      & 41        & 16\underline{8}1              & $8 \Rightarrow h(x)+1^2 = 8 + 1 = 9\text{ mod }10=9$  \\
Frank       & 25        & 6\underline{2}5               & $2 \Rightarrow h(x)+1^2 = 2 + 1 = 3\text{ mod }10=3 $    \\
Markus      & 32        & 10\underline{2}4              & $2 \Rightarrow h(x)+1^2 = 2 + 1 = 3 \Rightarrow h(x)+2^2= 2 + 4 = 6\text{ mod }10=6$ \\
Thomas      & 43        & 18\underline{4}9              & $4 \Rightarrow h(x)+1^2 = 4 + 1 = 5 \Rightarrow \dots \Rightarrow h(x)+6^2= 4 + 36 = 40\text{ mod }10=0$ \\
\hline
\end{tabular}

Somit ist die Hashtabelle wiefolgt befüllt:
\\

\begin{tabular}{|l|l|}
\hline
Behälter        & Name  \\
\hline
\hline
0 & Thomas \\
\hline
1 & -- \\
\hline
2 & Hartmut \\
\hline
3 & Frank \\
\hline
4 & Anna \\
\hline
5 & Christian \\
\hline
6 & Markus \\
\hline
7 & -- \\
\hline
8 & Heidrun \\
\hline
9 & Simone \\
\hline
\end{tabular}

\section*{EA3 - Aufgabe 2}
\section*{EA3 - Aufgabe 4}

\end{document}